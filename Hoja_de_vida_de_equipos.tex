\documentclass[a4paper, 10pt]{article}

% --- PAQUETES NECESARIOS ---
\usepackage[utf8]{inputenc}     % Permite usar caracteres en español como ñ, á, é, etc.
\usepackage[spanish]{babel}     % Configura el idioma del documento a español.
\usepackage{geometry}           % Para configurar los márgenes de la página.
\usepackage{amssymb}            % Provee el símbolo de la casilla de verificación (\square).
\usepackage{tabularx}           % Para crear tablas que se ajustan al ancho de la página.
\usepackage{array}              % Para mejorar la personalización de las tablas.
% --- CONFIGURACIÓN DE PÁGINA (AJUSTADA PARA UNA HOJA) ---
% Se reducen los márgenes superior e inferior para ganar espacio vertical.
\geometry{a4paper, left=1in, right=1in, top=0.8in, bottom=0.8in} 
% Se reduce el espaciado de las filas para que sea más compacto.
\renewcommand{\arraystretch}{1.6} 
\pagestyle{empty} % Elimina los números de página.
\begin{document}
% --- TÍTULO DEL DOCUMENTO ---
\begin{center}
    \Large\textbf{HOJA DE VIDA DE EQUIPOS}
\end{center}
% --- SECCIÓN 1: DATOS DEL EQUIPO ---
\subsection*{DATOS DEL EQUIPO}
\noindent % Evita la sangría al inicio del párrafo.
\begin{tabularx}{\textwidth}{|l|X|}
    \hline
    \textbf{Código} & \\
    \hline
    \textbf{Nombre} & \\
    \hline
    \textbf{Marca} & \\
    \hline
    \textbf{Modelo} & \\
    \hline
    \textbf{Fecha de compra} & \\
    \hline
    \textbf{Fecha de instalación} & \\
    \hline
    \textbf{Detalles técnicos} & \\
    \hline
\end{tabularx}
% --- SECCIÓN 2: MANTENIMIENTOS ---
\subsection*{Mantenimientos realizados}
\noindent
\begin{tabularx}{\textwidth}{|c|X|X|c|}
    \hline
    \textbf{Fecha} & \textbf{Descripción} & \textbf{Observaciones} & \textbf{Tipo de mmto} \\
    \hline
     & & & \\
    \hline
     & & & \\
    \hline
     & & & \\
    \hline
\end{tabularx}
% --- SECCIÓN 3: REPARACIONES MAYORES ---
\subsection*{Reparaciones mayores}
\noindent
\begin{tabularx}{\textwidth}{|c|X|c|X|}
    \hline
    % El título principal ahora abarca las 4 columnas.
    \multicolumn{4}{|c|}{\textbf{Reparaciones mayores}} \\
    \hline
    % Sub-encabezados para cada columna
    \textbf{P. reemplazadas} & \textbf{Descripción} & \textbf{Modificaciones} & \textbf{Descripción} \\
    \hline
     & & & \\
    \hline
     & & & \\
    \hline
     & & & \\
    \hline
\end{tabularx}
% --- SECCIÓN 4: RESPONSABLE ---
\subsection*{Responsable del equipo}
\noindent
\begin{tabularx}{\textwidth}{|l|X|}
    \hline
    \textbf{Nombre} & \\
    \hline
    \textbf{Cargo} & \\
    \hline
    \textbf{Dependencia} & \\
    \hline
\end{tabularx}
% --- SECCIÓN 5: ESTADO Y RETIRO ---
\subsection*{Estado y Retiro}
\noindent
\begin{tabularx}{\textwidth}{|p{0.5\textwidth}|X|}
    \hline
    \textbf{Estado actual} & \textbf{Fecha de retiro} \\
    \hline
    $\square$ En operación & \\
    \cline{1-1} % Línea solo para la primera columna.
    $\square$ Fuera de servicio & \\
    \cline{1-1}
    $\square$ En reparación & \\
    \hline
\end{tabularx}
\end{document}