\documentclass[12pt,letterpaper]{article}

% Paquetes necesarios para formato APA
\usepackage[utf8]{inputenc}
\usepackage[spanish]{babel}
\usepackage[margin=1in]{geometry}
\usepackage{setspace}
\usepackage{times}
\usepackage[style=apa,backend=biber]{biblatex}
\usepackage{csquotes}
\usepackage{fancyhdr}
\usepackage{titletoc}
\usepackage{tocloft}
\usepackage{amsmath}
\usepackage{amsfonts}
\usepackage{graphicx}
\usepackage{float}
\usepackage{caption}
\usepackage{subcaption}

% Configuración de la bibliografía
\addbibresource{lubri.bib}

% Configuración de página
\doublespacing
\setlength{\parindent}{0.5in}

% Configuración de encabezados
\pagestyle{fancy}
\fancyhf{}
\fancyhead[R]{\thepage}
\fancyhead[L]{LUBRICACIÓN Y MANTENIMIENTO}
\renewcommand{\headrulewidth}{0pt}
\setlength{\headheight}{15.14989pt}

% Configuración del título en tabla de contenido
\renewcommand{\contentsname}{Tabla de Contenido}
\renewcommand{\refname}{Referencias}

\begin{document}

% PORTADA
\begin{titlepage}
\centering

{\Large\textbf{TALLER FORMATOS 
LUBRICACIÓN Y MANTENIMIENTO }}\\[0.5cm]
{\large Orden de Mantenimiento y Hoja de Vida}\\[2cm]

{\large Presentado por:}\\[0.5cm]
{\large Gustavo Alejandro Vergara Pareja}\\%[2cm]

{\large Presentado a:}\\[0.5cm]
{\large Ing. Miguel Lancheros Montiel}\\[2cm]

{\large Universidad de Córdoba}\\
{\large Facultad de Ingeniería}\\
{\large Departamento de Ingeniería Mecánica}\\[5cm]

{\large Montería, Córdoba}\\
{\large \today}

\end{titlepage}

% TABLA DE CONTENIDO
\newpage
\tableofcontents
\newpage


\section{Orden de Mantenimiento}

Una orden de mantenimiento es un documento formal que inicia, planifica, controla y registra un trabajo de mantenimiento. Es la herramienta central para coordinar recursos, materiales y tiempo, asegurando que el trabajo se realice con calidad, seguridad y dentro del costo previsto.

\subsection{Funciones Principales}
\begin{itemize}
    \item Solicitar y autorizar trabajos.
    \item Planificar recursos (mano de obra, materiales, herramientas).
    \item Programar y priorizar tareas.
    \item Registrar costos, tiempos y materiales.
    \item Alimentar la hoja de vida del equipo.
\end{itemize}

\subsection{Contenido Mínimo}
\begin{itemize}
    \item Número de orden.
    \item Fecha de emisión.
    \item Departamento solicitante.
    \item Identificación del equipo (código, ubicación).
    \item Descripción del trabajo.
    \item Prioridad (emergencia, urgente, normal, programado).
    \item Recursos requeridos (personal, materiales, herramientas).
    \item Estándares de tiempo.
    \item Procedimientos de seguridad.
    \item Firma de aprobación y cierre.
\end{itemize}

\section{Hoja de Vida de un Equipo}


La hoja de vida de un equipo es el registro completo y sistemático de todos los datos técnicos y operativos relevantes desde la instalación hasta el retiro del activo. Este documento reúne información sobre intervenciones de mantenimiento (preventivo y correctivo), repuestos utilizados, tiempos empleados y costos asociados. Además, facilita el análisis de indicadores de confiabilidad como MTBF y MTTR, la planificación de mantenimientos y la toma de decisiones sobre reemplazo o mejora de equipos.\parencite{duffuaa2000}
\subsection{Contenido Mínimo}
\begin{itemize}
    \item Datos generales: código, nombre, ubicación, fabricante, modelo, serie y fecha de instalación.
    \item Especificaciones técnicas: capacidad, potencia, dimensiones.
    \item Fecha de instalación y puesta en marcha.
    \item Historial de mantenimiento: tipo (MP/MC), descripción, repuestos usados, costos, responsable, horas de parada y mediciones clave.
    \item Reparaciones y repuestos utilizados.
    \item Costos asociados.
    \item Observaciones, mejoras y evidencias de trabajos realizados.
\end{itemize}

\subsection{Buenas Prácticas}
\begin{itemize}
    \item Mantener la prioridad y el flujo de órdenes de trabajo: ¿cuál es la causa?, ¿planificación?, ¿asignación?, ¿ejecución?, ¿verificación?
    \item Usar estándares de tiempo y listas de verificación para asegurar la calidad y seguridad.
    \item Registrar todos los datos en sistemas informáticos para trazabilidad y control.
\end{itemize}

% Ejemplo de cita
Según Duffuaa et al. (2000), una gestión adecuada del mantenimiento es fundamental para la confiabilidad de los equipos.\parencite{duffuaa2000}



\newpage
\section{Referencias}
\printbibliography
\section{Anexos}
*Se incluyen en las siguientes páginas.
\end{document}